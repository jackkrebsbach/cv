%%%%%%%%%%%%%%%%%%%%%%%%%%%%%%%%%%%%%%%%%
% Medium Length Professional CV
% LaTeX Template
% Version 3.0 (December 17, 2022)
%
% This template originates from:
% https://www.LaTeXTemplates.com
%
% Author:
% Vel (vel@latextemplates.com)
%
% Original author:
% Trey Hunner (http://www.treyhunner.com/)
%
% License:
% CC BY-NC-SA 4.0 (https://creativecommons.org/licenses/by-nc-sa/4.0/)
%
%%%%%%%%%%%%%%%%%%%%%%%%%%%%%%%%%%%%%%%%%

%----------------------------------------------------------------------------------------
%	PACKAGES AND OTHER DOCUMENT CONFIGURATIONS
%----------------------------------------------------------------------------------------

\documentclass[
	%a4paper, % Uncomment for A4 paper size (default is US letter)
	11pt, % Default font size, can use 10pt, 11pt or 12pt
]{resume} % Use the resume class


\usepackage{ebgaramond} % Use the EB Garamond font


\name{Jackson Krebsbach} % Your name to appear at the top

% You can use the \address command up to 3 times for 3 different addresses or pieces of contact information
% Any new lines (\\) you use in the \address commands will be converted to symbols, so each address will appear as a single line.

\address{1270 Kuehnle Court, Ann Arbor, MI 48103} % Main address


\address{(734)~$\cdot$~678~$\cdot$~7984 \\ jacksonkrebsbach@gmail.com} % Contact information
\address{ www.github.com/jackkrebsbach }


%----------------------------------------------------------------------------------------

\begin{document}

%----------------------------------------------------------------------------------------
%	EDUCATION SECTION
%----------------------------------------------------------------------------------------

\begin{rSection}{Education}
	
	\textbf{Hope College} \hfill \textit{Expected May 2024} \\ 
	B.S. in Mathematics \\
	Concentration in Statistics \smallskip \\
	Overall GPA: 3.97

  \textbf{Highlighted Coursework}
  Statistics for Data Science, Advanced Linear Algebra, Real Analysis, Numerical Analysis, Introduction to Probability Algebraic Structures, Databases for Data Science, Software Design \& Implementation, Computer- Aided Design, Introduction to Mathematical Physics, Physics Lab: Electronics, Biomedical Instrumentation
	
\end{rSection}


\begin{rSection}{Awards and Honors}
  \begin{itemize}
    \item Pi Mu Epsilon Outstanding Speaker Award at Joint Mathematics Meeting \hfill \textit{January 2024}
    \item Erik Aasen Scholarship \hfill \textit{August 2022}

    \item Joint Mathematics Meetings Honourable Mention Poster \hfill \textit{January 2020}
    \item John H. Kleinheksel Mathematics Award \hfill \textit{May 2019} 
\item Pi Mu Epsilon Mathematics Honor Society Inductee \hfill \textit{May 2019} 
\item Hope College Presidential Scholarship \hfill \textit{August 2018} 

  \end{itemize}
    
\end{rSection}

\begin{rSection}{Grants}

  \begin{itemize}
    \item Pi Mu Epsilon Travel Grant, 2024 Joint Mathematics Meetings, \$1200

  \item Krebsbach, J., "Using Machine Learning and Drones to Estimate Vegetation Density in Coastal Sand Dunes," \$3,000. (May 10, 2020 - April 30, 2021). Funded by the National Aeronautics and Space Administration (NASA), under award number 80NSSC20M0124, Michigan Space Grant Consortium (MSGC)

  \item American Mathematical Society Travel Grant, 2020 Joint Mathematics Meetings, \$400
  \end{itemize}
\end{rSection}

%----------------------------------------------------------------------------------------
%	WORK EXPERIENCE SECTION
%----------------------------------------------------------------------------------------

\begin{rSection}{Experience}

%------------------------------------------------

%------------------------------------------------

  \begin{rSubsection}{Undergraduate Researcher}{May 2019 - Present}{Hope College Mathematics \& Statistics Department}{Holland, MI}
  \item Main work consists of using machine learning and unmanned aerial systems to map surface composition in Lake Michigan sand dunes
  \item Conducted field work, flying drones to capture multi-spectral imagery and acquire  ground-based photography  at Saugatuck Harbor Natural Area.
  \item Performed big data analysis in R and Python, generating feature imagery, sampling training data, and training machine learning algorithms
  \item Gave numerous talks and presented several posters at Join Mathematics Meetings, Geological Society Association Meeting, Pi Mu Epsilon Meetings, Hope College colloquium, and Mathfest
	\end{rSubsection}


\pagebreak
	\begin{rSubsection}{Teaching Assistant}{ January 2024 - Present}{Hope College}{Holland, MI}
    \item Assisting students in lab and course material for accelerated statistics (Math 219) and statistics for data science (Math 313)
    \item Grade lab assignments completed in R
	\end{rSubsection}

	\begin{rSubsection}{Accenture Student Consultant}{ January 2024 - Present}{Hope College Center for Leadership}{Holland, MI}
    \item Student consultant project for Accenture IT company 
    \item Research and ranking of the Top 3-5 SC Analytics Platforms including a comparative analysis
    \item Demystifying the generative AI vs traditional AI components
	\end{rSubsection}


	\begin{rSubsection}{REZA INC.}{May 2019 - Present}{Co-Founder}{Detroit, MI}
  \item Co-created REZA INC., a VC backed light-up footwear
brand dedicated to inspiring people to ‘Light Your Own
Path’
\item Completed residency at Techstars Sports Accelerator Powered by Indy (2020)\item Sourced components and completed shoe development in Taiwan (Nov 2020 – Apr 2021, March 2023 – June 2023)
\item Sold over 2,000 pairs and acquired wait list of 70K+
	\end{rSubsection}


	\begin{rSubsection}{Ford Motor Company}{June 2022 - August 2022}{Internship}{Dearborn, MI}

  \item Worked as an intern in software product development  
  \item End to end data pipeline sourced from features in vehicles to create data visualizations using SQL, Python, Putty, Amplitude.
  \item  Created a clinic to evaluate the digital owner's manual found in the entertainment system in the Ford F150 Lightning
  \item Presented recommendations to executives based on insights gained from study
	\end{rSubsection}

	\begin{rSubsection}{Software Developer}{July 2023 - September 2023 }{Contractor}{Remote}
  \item Mobile app development for a venture capital backed stealth social media start-up
  \item Assisted and interviewed candidates for full time roles
  \item Worked with a team consisting of one backend developer
and two front end engineers
\item Technologies: Firebase, React Native, Test Flight

	\end{rSubsection}

  \begin{rSubsection}{ Mathematics and Computer Science Tutor}{August 2019 - May 2020}{Hope College Academic Success Centre}{Holland, MI}
    \item Hired as a tutor for the Software Design \& Implementation CS course using the Java programming language
    \item Led group of four students through the Fall semester of 2019 assisting in course material and projects
    \item Worked in the Hope College Math Lab for lower and upper-level mathematics courses in the Spring of 2020
    \item  Provided mathematical guidance to students on an individual and group basis in help sessions.
	\end{rSubsection}

  \begin{rSubsection}{Youth Ambassador to the Philippines}{July 2016 - April 2017}{U.S State Department}{ Biñan, Philippines}
  \item High-school exchange student in the Philippines  supported by the Yes-Abroad Kennedy-Lugar Scholarship program
  \item Lived with a Filipino family for a period 10-months in Biñan, Laguna
  \item Studied at Jacobo national high school and University of Perpetual Help
	\end{rSubsection}

\end{rSection}

\pagebreak

%----------------------------------------------------------------------------------------
%	TECHNICAL STRENGTHS SECTION
%----------------------------------------------------------------------------------------

\begin{rSection}{Technical Strengths}

	\begin{tabular}{@{} >{\bfseries}l @{\hspace{6ex}} l @{}}
		Computer Languages & R, Python, Jupyter, MATLAB, JavaScript, Typescript, Java, HTML, SCSS, CSS
 \\
    Technologies \& Frameworks & Git, Vim, Linux, SQL, QGIS, Node.js, React, Next.js, React Native,  \\ & RStudio, Jupyter, Autodesk Inventor, Agisoft Metashape \\ &  Vercel, Google Analytics, Shopify,

	\end{tabular}

\end{rSection}

%----------------------------------------------------------------------------------------
%	EXAMPLE SECTION
%----------------------------------------------------------------------------------------

\begin{rSection}{Publications in Preparation}
    
\item Krebsbach, J., Yurk, B. P., DeVries-Zimmerman, S. J., Pearson, P., Hansen, E. C. “Mapping vegetation in Lake Michigan sand dunes using unoccupied aerial systems and machine learning” \textit{In Preparation}
\end{rSection}

\begin{rSection}{Selected Presentations}

  \begin{itemize}
      
\item Krebsbach, J., (Yurk, B. P., Mentor), Joint Mathematics Meeting, Talk, “Mapping Plant Populations Using Drones and Machine Learning”, San Francisco, CA. (January 4, 2024)
\item Krebsbach, J., (Yurk, B. P., Mentor), 48th Annual Pi Mu Epsilon Conference, Talk, “Mapping Vegetation in Lake Michigan Dunes with XGBoost”, Miami, OH. (September 29, 2023)

\item Krebsbach, J., Yurk, B. P., DeVries-Zimmerman, S. J., Pearson, P., Hansen, E. C. International Conference on Aeolian Research, Poster. “Mapping vegetation in Lake Michigan sand dunes using unoccupied aerial systems and machine learning,” Las Cruces, NM. (July 13, 2023)

\item Krebsbach, J., Yurk, B. P. Mathfest, 2021, Talk, “Modeling Vegetation Density,” Online. (August, 5, 2021)

\item Krebsbach, J. (Yurk, B. P., Mentor). Midstates Consortium for Math and Science Undergraduate Research Symposium, Talk, "Dunes \& Drones: Using machine learning to map vegetation with drone- and ground-based photography," Online. (November 7, 2020)

\item Krebsbach, J., Yurk, B. P. Joint Mathematics Meeting, Poster, "Mapping dune vegetation using drones, ground photography, and machine learning," Denver, CO. (January 17, 2020).


\item Krebsbach, J., Yurk, B. P., Pearson, P. T., Stid, J., Hansen, E. C. Geological Society of America Annual Meeting, Poster, "Vegetation and Topography Mapping of Coastal Dune Complexes Using Small Unmanned Aerial Systems and Ground-Based Imagery,” Phoenix, AZ. (September 22, 2019)


\item Krebsbach, J., (Yurk, B. P., Mentor), PME Mathfest, Talk, “Dunes and drones: A machine learning approach to mapping dune vegetation using small unmanned aerial systems and ground based photography, Cincinnati, OH. (August 1st, 2019)
  \end{itemize}

\end{rSection}


\begin{rSection}{Media}

  \textbf{Grand Rapids Magazine} https://www.grmag.com/look-feel/style/fresh-kicks-bright-future/ \hfill \textit{November 2023}
    
\end{rSection}


%----------------------------------------------------------------------------------------

\end{document}

